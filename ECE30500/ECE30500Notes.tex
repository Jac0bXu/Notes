\documentclass[nobib]{tufte-handout}

%\\geometry{showframe}% for debugging purposes -- displays the margins

\newcommand{\bra}[1]{\left(#1\right)}
\usepackage{hyperref}
\usepackage[activate={true,nocompatibility},final,tracking=true,kerning=true,spacing=true,factor=1100,stretch=10,shrink=10]{microtype}
\usepackage{color}

% Fixes captions and images being cut off
\usepackage{marginfix}

\usepackage{pgfplots}
\usepackage[americancurrents, americaninductors, americanvoltages, americanresistors]{circuitikz}
\usepackage{siunitx}
\usepackage{amsmath,amsthm}
\usetikzlibrary{shapes}
\usetikzlibrary{positioning}
\usetikzlibrary{arrows}

% Set up the images/graphics package
\usepackage{graphicx}
\setkeys{Gin}{width=\linewidth,totalheight=\textheight,keepaspectratio}
\graphicspath{{.}}

\title{Notes for ECE 30500 - Semiconductor Devices}
\author[Shubham Saluja Kumar Agarwal]{Shubham Saluja Kumar Agarwal}
\date{\today}  % if the \date{} command is left out, the current date will be used

% The following package makes prettier tables.  We're all about the bling!
\usepackage{booktabs}

% The fancyvrb package lets us customize the formatting of verbatim
% environments.  We use a slightly smaller font.
\usepackage{fancyvrb}
\fvset{fontsize=\normalsize}

% Small sections of multiple columns
\usepackage{multicol}

% These commands are used to pretty-print LaTeX commands
\newcommand{\doccmd}[1]{\texttt{\textbackslash#1}}% command name -- adds backslash automatically
\newcommand{\docopt}[1]{\ensuremath{\langle}\textrm{\textit{#1}}\ensuremath{\rangle}}% optional command argument
\newcommand{\docarg}[1]{\textrm{\textit{#1}}}% (required) command argument
\newenvironment{docspec}{\begin{quote}\noindent}{\end{quote}}% command specification environment
\newcommand{\docenv}[1]{\textsf{#1}}% environment name
\newcommand{\docpkg}[1]{\texttt{#1}}% package name
\newcommand{\doccls}[1]{\texttt{#1}}% document class name
\newcommand{\docclsopt}[1]{\texttt{#1}}% document class option name

% Define a custom command for definitions
\newcommand{\defn}[2]{\noindent\textbf{#1}:\ #2}

\begin{document}

\maketitle

\begin{abstract}
    These are lecture notes for Fall 2025 ECE 30500 by professor Haitong Li at Purdue. Modify, use, and distribute as you please.
\end{abstract}

\tableofcontents

\newpage

\section{Properties of Silicon}
The core of semiconductors lies in the silicon transistor. But, why silicon (\textbf{Si})?\\
\begin{itemize}
    \item Si is the second most common element on Earth.
    \item It is easily purified, and grown defect free, with less than 1 impurity in $10^9$ atoms.
    \item Reasonably good electronic properties
    \item Resilient to harsh environments
    \item Excellent mechanical properties
    \item There are three forms of Si: \begin{itemize}
        \item In a Si crystal, atoms are arranged in an orderly array, allowing arrangements to be easily reproduced.
        \item In poly-crystalline Si, many crystalline subsections exist.
        \item In amorphous Si, there are no long range patterns or arrangements.
    \end{itemize}
\end{itemize}
The unit cell is a portion of any crystal that could be used to reproduce the crystal.\\
The primitive cell is the smallest possible unit cell.\\
In essence, a unit cell is a subset of a lattice that can be moved in the x, y, z axis, and cover the entire lattice. (\textit{Note the absence of rotation movements.})\\
Some examples of cells are the following:
\begin{center}
    \includegraphics[width = 250px]{images/unit_cell_cube.png}
\end{center}
\textit{Note: the image is missing a corner atom.}\\
Another important cubic unit cell is the diamond cubic unit cell with 8 silicon atoms in the cell:
\begin{center}
    \includegraphics[width = 175px]{images/diamond_cube_cell.png}
\end{center}
\subsection{Density (diamond cube cell)}
Lattice constant: $a=5.3407Ang$\\
Atomic mass: $28.055$amu\\
Density: $\rho = \frac{8*28.0855*1.6605*10^{-23}}{(5.4307*10^{-10})^3}kg/m^3 = 2.3296g/cm^3$
\subsection{Miller Indices}
Let us consider a plane that intercepts the axes at $x_{int}, y_{int}, z_{int}$.
The equation of the plane is:
\begin{equation*}
    \frac{x}{x_{int}}+\frac{y}{y_{int}}+\frac{z}{z_{int}} = 1
\end{equation*}
The vector that is perpendicular to this plane will have the same components as the Miller indices.\\
The Miller indices are defined as $LCM*(\frac{1}{x_{int}},\frac{1}{y_{int}},\frac{1}{z_{int}})$.
\section{Energy Bands}
\subsection{Quantization of Energy Levels}
Bohr hypothesized, in his atomic model, that there was a quantization of electron angular momentum and energy levels.\\
In the context of this class, these levels will be described by $n \in \mathbb{N}$.\\
Additionally, the energy of these levels is $E_H = -\frac{13.6}{n^2}eV$.\\
The energy levels of silicon are as shown below:\\
\begin{center}
    \includegraphics*[height = 100px]{images/si_energy_levels.png}
\end{center}
The further from the core, the more energy is within, so, in the image above, $4S^0$ is the layer with the most energy.\\
The four electrons in the valence region ($3P^2,3S^2$) are the easiest to break away from the atom.
\subsection{Bonding Model}
Only the four valence electrons are of interest. Each of these, is shared with one of the four nearest neighbors, forming a structure like this:
\begin{center}
    \includegraphics*[width = 100px]{images/silicon_neighbor_bond.png}
\end{center}
This is the base of the silicon lattice, from which more complex lattices can be formed. The full model can be abstracted as follows:
\begin{center}
    \includegraphics*[width = 100px]{images/bonding_model.png}
\end{center}
where a line is a shared valence electron and a circle is the core of the semiconductor.\\
\subsection{Energy Levels}
When going from a single Si atom to a Si crystal with many atoms, the atoms are closely packed enough that we cannot treat them as individual atoms. The structure that is formed is very stable as all the atoms and their energy levels are completely filled.\\
\textbf{Note}: Current cannot flow from full energy levels.\\
The energy levels of these atoms are in the following order:
\begin{eqnarray*}
    &\text{conduction band}\\
    &\cdots \cdots \cdots \\
    &\uparrow \\
    &\text{band gap}\\
    &\uparrow \\
    &\cdots \cdots \cdots \\
    &\text{valence band}
\end{eqnarray*}
The conduction band and the valence band are energy levels where electrons can reside.\\
Electrons can be freed from their bonds by heating the lattice to temperatures close to $300K$. The thermal energy provided by this is defined by:
\begin{equation*}
    E = \frac{3}{2}kT
\end{equation*}
The freed electrons are analyzed in terms of their energy states. When freed, they are moving from the valence band (leaving behind a "hole" in the valence band), jumping over the band gap and arriving at the conduction band.\\
The band representation can be further abstracted to look like this:
\begin{center}
    \includegraphics*[width = 100px]{images/band_diagram_abstraction.png}
\end{center}
Where the red dots are conduction band electrons (n) and the blue dots are valence band holes (p).
\textbf{Note}: an intrinsic semiconductor is that in which $n = p = n_i$.
\subsection*{Constants to Remember}
\begin{align*}
    E_G(Si) &= 1.12eV \\
    E_G(GaAs) &= 1.4eV \\
    k_B T &= 0.026eV \text{ at }(T = 300K)\\
    P &\approx e^{-E_G / k_B T}\\
    n_i(Si) &= 1\times 10^{10} cm^{-3} \text{ at }(T = 300K)\\
    n_i(GaAs) &= 2\times 10^{6} cm^{-3} \text{ at }(T = 300K)\\
\end{align*}
\subsection*{Material Representations}
\begin{center}
    \includegraphics*[width = 250px]{images/material_representations.png}
\end{center}
Note that the main difference between an insulator and a semiconductor is the size of the band gap. On the other hand, metals have no band gap at all, and all the electrons are free by default.
\end{document}
