\documentclass[nobib]{tufte-handout}

%\\geometry{showframe}% for debugging purposes -- displays the margins

\newcommand{\bra}[1]{\left(#1\right)}
\usepackage{hyperref}
\usepackage[activate={true,nocompatibility},final,tracking=true,kerning=true,spacing=true,factor=1100,stretch=10,shrink=10]{microtype}
\usepackage{color}

% Fixes captions and images being cut off
\usepackage{marginfix}

\usepackage{pgfplots}
\usepackage[americancurrents, americaninductors, americanvoltages, americanresistors]{circuitikz}
\usepackage{siunitx}
\usepackage{amsmath,amsthm}
\usetikzlibrary{shapes}
\usetikzlibrary{positioning}
\usetikzlibrary{arrows}

% Set up the images/graphics package
\usepackage{graphicx}
\setkeys{Gin}{width=\linewidth,totalheight=\textheight,keepaspectratio}
\graphicspath{{.}}

\title{Notes for ECE 30200 - Probabilistic Methods in Electrical and Computer Engineering}
\author[Shubham Saluja Kumar Agarwal]{Shubham Saluja Kumar Agarwal}
\date{\today}  % if the \date{} command is left out, the current date will be used

% The following package makes prettier tables.  We're all about the bling!
\usepackage{booktabs}

% The fancyvrb package lets us customize the formatting of verbatim
% environments.  We use a slightly smaller font.
\usepackage{fancyvrb}
\fvset{fontsize=\normalsize}

% Small sections of multiple columns
\usepackage{multicol}
\usepackage{mdframed}
% These commands are used to pretty-print LaTeX commands
\newcommand{\doccmd}[1]{\texttt{\textbackslash#1}}% command name -- adds backslash automatically
\newcommand{\docopt}[1]{\ensuremath{\langle}\textrm{\textit{#1}}\ensuremath{\rangle}}% optional command argument
\newcommand{\docarg}[1]{\textrm{\textit{#1}}}% (required) command argument
\newenvironment{docspec}{\begin{quote}\noindent}{\end{quote}}% command specification environment
\newcommand{\docenv}[1]{\textsf{#1}}% environment name
\newcommand{\docpkg}[1]{\texttt{#1}}% package name
\newcommand{\doccls}[1]{\texttt{#1}}% document class name
\newcommand{\docclsopt}[1]{\texttt{#1}}% document class option name

% Define a custom command for definitions

\newmdenv[
    backgroundcolor=black!10, % Light blue shading inside the box
    linecolor=black,         % Border color
    linewidth=1pt,          % Border thickness
    roundcorner=5pt,        % Rounded corners
    skipabove=10pt,         % Space above the box
    skipbelow=10pt,         % Space below the box
    innertopmargin=5pt,     % Inner top margin
    innerbottommargin=5pt,  % Inner bottom margin
    innerleftmargin=10pt,   % Inner left margin
    innerrightmargin=10pt   % Inner right margin
    ]{defbox}

\newmdenv[
    backgroundcolor=cyan!10, % Light blue shading inside the box
    linecolor=cyan,         % Border color
    linewidth=1pt,          % Border thickness
    roundcorner=5pt,        % Rounded corners
    skipabove=10pt,         % Space above the box
    skipbelow=10pt,         % Space below the box
    innertopmargin=5pt,     % Inner top margin
    innerbottommargin=5pt,  % Inner bottom margin
    innerleftmargin=10pt,   % Inner left margin
    innerrightmargin=10pt   % Inner right margin
    ]{notebox}
    
\newcommand{\defn}[2]{
        \begin{defbox}
        \noindent\textbf{#1}:\ #2
        \end{defbox}
}
\newcommand{\note}[1]{
        \begin{notebox}
        \noindent\textit{Note}:\ #1
        \end{notebox}
}

\begin{document}

\maketitle

\begin{abstract}
    These are lecture notes for Fall 2025 ECE 30200 by professor Mary Comer at Purdue. Modify, use, and distribute as you please.
\end{abstract}

\tableofcontents

\newpage

\section{Modelling Random Experiments}
Examples:
\begin{itemize}
    \item Flip a coin 
    \item Rolling dice
    \item Generate a bit from a random binary source
    \item Generate a sequence of $n$ bits from a random binary source
    \item Count packets arriving at a router
    \item Measure the voltage at a point in a circuit
\end{itemize}
All of these examples could be at a fixed time, or over an interval of time.\\~\\
There is a very precise framework that can be used to model any random experiment.\\
Overview:
\begin{itemize}
    \item The outcome that occurs each time a random experiment is run is not known in advance, but \textbf{the set of all possible outcomes is assumed to be known}.
    \item Subsets of the set of all possible outcomes are called events.
    \item Probabilities are assigned to events, not outcomes, using a probability measure.
\end{itemize}
We need to use set theory to work with these sets.
\section{Set Theory}
A set is an \textbf{unordered} collection of elements denoted by $\{ \}$.\\
Example:
\begin{equation*}
    \{1,2,3\} = \{3,2,1\} = \{2,1,3,1\}
\end{equation*}
Notation:
\begin{itemize}
    \item $w \in A$ means $w$ is in set $A$
    \item $w \not\in A$ means $w$ is not in set $A$
\end{itemize}
There are two ways to specify a set:
\begin{enumerate}
    \item Comma-separated list:
    \begin{itemize}
        \item $A = \{1,3,A,C\}$
        \item $A = \{ x_1, x_2, \ldots, x_n\}$ (that is, $n$ elements for some finite $n \geq 1$ and the $i$th element is $x_i$)
        \item $A = \{x_1,x_2, \ldots\}$ (an infinite number of elements)
    \end{itemize}
    \item  Rule for membership or set builder notation:
    \begin{itemize}
        \item $A = \{w \in \mathbb{Z} : 1\leq w \leq 6\}$
        \item Special notation for intervals $\in \mathbb{R}$:
        \begin{itemize}
            \item $(a,b)$ represents $\{x \in \mathbb{R}: a < x < b\}$
            \item $[a,b]$ represents $\{x \in \mathbb{R}: a \leq x \leq b\}$
        \end{itemize}
        \textit{Note}: $(a,b) \subset \mathbb{R}$ is a set, but $(a,b) \in \mathbb{R}^2 = \mathbb{R} \times \mathbb{R}$ is also a point on the Cartesian plane or an ordered pair.
    \end{itemize}
\end{enumerate}
\defn{Equal sets}{sets $A$ and $B$ are equal if they contain exactly the same elements, and is denoted by $A=B$.}
\defn{$\mathbf{A \subset B}$}{$(w\in A \implies w\in B)\implies (A \subset B)$ for sets $A$ and $B$ where $A \subset B$ means $A$ is a subset of $B$.}
\note{ we will not distinguish between proper subsets: 
\begin{equation*}
    (A\subset B) \&( A \neq B)
\end{equation*}
and subsets.}
\begin{equation*}
    A = B \iff (A \subset B) \& (B \subset A)
\end{equation*}
\subsection{Special Sets}
The set with no elements is called the empty set or the null set.\\
Denoted by $\emptyset \text{ or } \{\}$ which is not the same as $\{\emptyset\}$.\\~\\
The set containing all possible elements of interest is called the universal set $S$. It is known as the sample space within probability, and will contain all possible outcomes of a random experiment.\\
\subsection{Set Operations}
\begin{itemize}
    \item Intersect ($\cap$): $A\cap B = \{w \in S: (w \in A) \& (w \in B) \}$
    \item Union ($\cup$): $A\cap B = \{w \in S: (w \in A) | (w\in B)\}$
    \item Complement ($\fbox{\phantom{}}^c$, $\fbox{\phantom{}}'$, $\bar{\fbox{\phantom{}}}$): $A^c = \{w \in S: w \not \in A\}$
    \item Disjoint: $A\cap B = \emptyset$
    \item \textit{Difference}: $A-B = A\cap B^c$
\end{itemize}
\subsection{Set Algebra}
Set algebra provides us with a second method to prove that two sets $A,B$ are equal.\\
It uses the fact that both union and intersection are commutative and associative to prove the statement.\\
The following are a few properties that come from commutativity and associativity:
\begin{itemize}
    \item $A\cup B = B\cup A$
    \item $A\cap B = B\cap A$
    \item $(A\cap B)\cap C = A \cap(B\cap C)$
    \item $(A\cup B)\cup C = A \cup(B\cup C)$
\end{itemize}
$\cup$ is distributive over $\cap$ and vice versa:
\begin{itemize}
    \item $A \cap (B\cup C) = (A\cap B) \cup (A\cap C)$
    \item $A \cup (B\cap C) = (A\cup B) \cap (A\cup C)$
\end{itemize}
\note{ when translating between Set Theory and English
    \begin{itemize}
        \item $\cup = $ OR  
        \item $\cap = $ AND  
    \end{itemize}
}
\subsection{Venn Diagram}
A Venn diagram is a graphical representation of a universal set and its subsets.
For example:
\begin{center}
    \includegraphics*[width = 100px]{images/venn_example.png}
\end{center}

\note{ A Venn diagram does not represent arbitrary sets, and thus, is not a proof by itself.}

\subsection{Set Types}
There are three kinds of sets:
\begin{enumerate}
    \item Finite: a set is finite if it has a finite number of elements.
    \item Countably Infinite: a set is countably infinite if it can be placed into one-to-one correspondance with the integers. Can be written as \begin{equation*}
        A = \{x_1,x_2,\ldots\} \text{ where } x_i \neq x_j \forall i \neq j
    \end{equation*}
    \item Uncountable: set that is neither countable nor finite. 
    \note{For the context of this class, the uncountably infinite sets will only be $\mathbb{R}$ or intervals of the same.}
    They cannot be written as $\{x_1,x_2,\ldots\}$
\end{enumerate}
\subsection{Collections of Sets}
\begin{itemize}
    \item Finite Collections:
            \begin{equation*}
                A_1, A_2, \ldots, A_n
            \end{equation*}
            where $A_i \subset S$
    \item Countably Infinite Collection:
            \begin{equation*}
                A_1, A_2, \ldots
            \end{equation*}
            where $A_i \subset S$
    \item Uncountable Collection: out of bounds for this class!
\end{itemize}
\defn{Union of collection}{$\bigcup_i^{\infty,n} A_i = \{w \in S$: $\exists i | w \in A_i$\}}
\defn{Intersection of collection}{$\bigcap_i^{\infty,n} A_i = \{w \in S$: $w \in A_i \forall i$\}}

\section{Modelling Random Experiments Pt.2}
We use a probability space to model (almost) any random experiment.
Three parts make up a probability space: 
\begin{enumerate}
    \item Sample Space (S): nonempty set of elements called outcomes. Each time an experiment is run, exactly one outcome occurs.
    \item 
\end{enumerate}
\end{document}
