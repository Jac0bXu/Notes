\documentclass[nobib]{tufte-handout}

%\\geometry{showframe}% for debugging purposes -- displays the margins

\newcommand{\bra}[1]{\left(#1\right)}
\usepackage{hyperref}
\usepackage[activate={true,nocompatibility},final,tracking=true,kerning=true,spacing=true,factor=1100,stretch=10,shrink=10]{microtype}
\usepackage{color}

% Fixes captions and images being cut off
\usepackage{marginfix}

\usepackage{pgfplots}
\usepackage[americancurrents, americaninductors, americanvoltages, americanresistors]{circuitikz}
\usepackage{siunitx}
\usepackage{amsmath,amsthm}
\usetikzlibrary{shapes}
\usetikzlibrary{positioning}
\usetikzlibrary{arrows}

% Set up the images/graphics package
\usepackage{graphicx}
\setkeys{Gin}{width=\linewidth,totalheight=\textheight,keepaspectratio}
\graphicspath{{.}}

\title{Notes for ECE 30200 - Probabilistic Methods in Electrical and Computer Engineering}
\author[Shubham Saluja Kumar Agarwal]{Shubham Saluja Kumar Agarwal}
\date{\today}  % if the \date{} command is left out, the current date will be used

% The following package makes prettier tables.  We're all about the bling!
\usepackage{booktabs}

% The fancyvrb package lets us customize the formatting of verbatim
% environments.  We use a slightly smaller font.
\usepackage{fancyvrb}
\fvset{fontsize=\normalsize}

% Small sections of multiple columns
\usepackage{multicol}

% These commands are used to pretty-print LaTeX commands
\newcommand{\doccmd}[1]{\texttt{\textbackslash#1}}% command name -- adds backslash automatically
\newcommand{\docopt}[1]{\ensuremath{\langle}\textrm{\textit{#1}}\ensuremath{\rangle}}% optional command argument
\newcommand{\docarg}[1]{\textrm{\textit{#1}}}% (required) command argument
\newenvironment{docspec}{\begin{quote}\noindent}{\end{quote}}% command specification environment
\newcommand{\docenv}[1]{\textsf{#1}}% environment name
\newcommand{\docpkg}[1]{\texttt{#1}}% package name
\newcommand{\doccls}[1]{\texttt{#1}}% document class name
\newcommand{\docclsopt}[1]{\texttt{#1}}% document class option name

% Define a custom command for definitions
\newcommand{\defn}[2]{\noindent\textbf{#1}:\ #2}

\begin{document}

\maketitle

\begin{abstract}
    These are lecture notes for Fall 2025 ECE 30200 by professor Mary Comer at Purdue. Modify, use, and distribute as you please.
\end{abstract}

\tableofcontents

\newpage

\section{Modelling Random Experiments}
Examples:
\begin{itemize}
    \item Flip a coin 
    \item Rolling dice
    \item Generate a bit from a random binary source
    \item Generate a sequence of $n$ bits from a random binary source
    \item Count packets arriving at a router
    \item Measure the voltage at a point in a circuit
\end{itemize}
All of these examples could be at a fixed time, or over an interval of time.\\~\\
There is a very precise framework that can be used to model any random experiment.\\
Overview:
\begin{itemize}
    \item The outcome that occurs each time a random experiment is run is not known in advance, but \textbf{the set of all possible outcomes is assumed to be known}.
    \item Subsets of the set of all possible outcomes are called events.
    \item Probabilities are assigned to events, not outcomes, using a probability measure.
\end{itemize}
We need to use set theory to work with these sets.
\section{Set Theory}
A set is an \textbf{unordered} collection of elements denoted by \{ \}.\\
Example:
\begin{equation*}
    \{1,2,3\} = \{3,2,1\} = \{2,1,3,1\}
\end{equation*}
Notation:
\begin{itemize}
    \item $w \in A$ means $w$ is in set $A$
    \item $w \not\in A$ means $w$ is not in set $A$
\end{itemize}
\end{document}
