\documentclass[nobib]{tufte-handout}

%\\geometry{showframe}% for debugging purposes -- displays the margins

\newcommand{\bra}[1]{\left(#1\right)}
\usepackage{hyperref}
\usepackage[activate={true,nocompatibility},final,tracking=true,kerning=true,spacing=true,factor=1100,stretch=10,shrink=10]{microtype}
\usepackage{color}

% Fixes captions and images being cut off
\usepackage{marginfix}
\usepackage{listings}


\usepackage{pgfplots}
\usepackage[americancurrents, americaninductors, americanvoltages, americanresistors]{circuitikz}
\usepackage{siunitx}
\usepackage{amsmath,amsthm}
\usetikzlibrary{shapes}
\usetikzlibrary{positioning}
\usetikzlibrary{arrows}

% Set up the images/graphics package
\usepackage{graphicx}
\setkeys{Gin}{width=\linewidth,totalheight=\textheight,keepaspectratio}
\graphicspath{{.}}

\title{Notes for ECE 20875 - Python for Data Science}
\author[Shubham Saluja Kumar Agarwal]{Shubham Saluja Kumar Agarwal}
\date{\today}  % if the \date{} command is left out, the current date will be used

% The following package makes prettier tables.  We're all about the bling!
\usepackage{booktabs}

% The fancyvrb package lets us customize the formatting of verbatim
% environments.  We use a slightly smaller font.
\usepackage{fancyvrb}
\fvset{fontsize=\normalsize}

% Small sections of multiple columns
\usepackage{multicol}

% These commands are used to pretty-print LaTeX commands
\newcommand{\doccmd}[1]{\texttt{\textbackslash#1}}% command name -- adds backslash automatically
\newcommand{\docopt}[1]{\ensuremath{\langle}\textrm{\textit{#1}}\ensuremath{\rangle}}% optional command argument
\newcommand{\docarg}[1]{\textrm{\textit{#1}}}% (required) command argument
\newenvironment{docspec}{\begin{quote}\noindent}{\end{quote}}% command specification environment
\newcommand{\docenv}[1]{\textsf{#1}}% environment name
\newcommand{\docpkg}[1]{\texttt{#1}}% package name
\newcommand{\doccls}[1]{\texttt{#1}}% document class name
\newcommand{\docclsopt}[1]{\texttt{#1}}% document class option name

% Define a custom command for definitions
\newcommand{\defn}[2]{\noindent\textbf{#1}:\ #2}

\begin{document}

\maketitle

\begin{abstract}
    These are lecture notes for Fall 2025 ECE 20875 by professor Aristides Carrillo at Purdue. Modify, use, and distribute as you please.
\end{abstract}

\tableofcontents

\newpage

\section{Introduction}
Data has a multitude of definitions, but it originated from information, and is
used to generate something, be it knowledge or beliefs. There are also several
kinds of data, such as quantitative and qualitative, or physical and digital.\\
Data analysis helps us make decisions and take actions.\\ Data science has
several branches, such as collecting and organizing data, making observations,
visualizing trends, identifying similarities, making predictions, prescribing
courses of action, and developing and accelerating algorithms.\\

\section{Histograms}
\textbf{Count}: number of elements in each bin of the histogram, and is denoted by $x_k$.
\begin{equation*}
    \sum_{k=1}^{n}x_k = m
\end{equation*}
where the sum of all bins is the total number of samples $m$.\\
\textbf{Probability}: probability of the occurrence of each bin, and is denoted by $\hat{p_k} = \frac{x_k}{\sum_{l}x_l}$.
\begin{equation*}
    \sum_{k}\hat{p_k} = 1
\end{equation*}
that is, the sum of all probabilities is 1.\\
\textbf{Density}: normalization of probability and bin width, denoted by $\hat{d_k} = \frac{\hat{p_k}}{w}$.
\begin{equation*}
    \sum_{k}w\hat{d_k} = 1
\end{equation*}
that is, the area under the probability curve is 1.\\
\textit{Note}: the term "frequency" can be applied to both "count" and "probability".
\subsection{Number of Bins}
There are several parameters defining a histogram, including number of bins
$n$, the width $w$, and number of samples $m$.\\ Bins do not need to all have
the same width.\\ The selection of $n$ and $w$ can be done through a number of
methods:
\begin{itemize}
    \item $n = \sqrt{m}$
    \item $n = \lceil \log m \rceil + 1$
    \item $n = \lceil 2m^{1/3} \rceil$
    \item $w = 3.5 \frac{\hat{\sigma}}{m^{1/3}}$
\end{itemize}
All of these methods have different considerations.\\
$w$ should be selected to minimize the error of estimating a point.
The Integrated Square Error (ISE), or the smoothing parameter is:
\begin{equation*}
    L(w) = \int (\hat{f_m}(x)-f(x)^2)dx
\end{equation*}
where $\hat{f_m}(x)$ is the density estimate of the histogram with $m$ samples, while $f(x)$ is the true but unknown model.\\
Since there are unknowns in the above equation, we estimate as follows:
\begin{equation*}
    L(w)\approx J(w) + K = \frac{2}{(m-1)w}-\frac{m+1}{(m-1)w}(\hat{p_1}^2+\hat{p_2}^2+\cdots+\hat{p_n}^2) +K
\end{equation*}
where $\hat{p_k}$ are the individual bin probabilities and $K$ is a constant. \\
Bin width is optimized by selecting the value that minimizes $J(w)$. This can be approached through brute force methods. However, since $n$ is finite, it can be better to run through values of $n$ instead (using grid search).
\section{Probability}
Conduct an experiment, get an outcome.
The outcome has a probability of occurring between 0 and 1.\\
The set of all possible outcomes is the sample space.\\
Sum of all probabilities of all outcomes is 1.\\
An event is a set of possible outcomes.\\

\end{document}
