\documentclass[nobib]{tufte-handout}

%\\geometry{showframe}% for debugging purposes -- displays the margins

\newcommand{\bra}[1]{\left(#1\right)}
\usepackage{hyperref}
\usepackage[activate={true,nocompatibility},final,tracking=true,kerning=true,spacing=true,factor=1100,stretch=10,shrink=10]{microtype}
\usepackage{color}

% Fixes captions and images being cut off
\usepackage{marginfix}

\usepackage{pgfplots}
\usepackage[americancurrents, americaninductors, americanvoltages, americanresistors]{circuitikz}
\usepackage{siunitx}
\usepackage{amsmath,amsthm}
\usetikzlibrary{shapes}
\usetikzlibrary{positioning}
\usetikzlibrary{arrows}

% Set up the images/graphics package
\usepackage{graphicx}
\setkeys{Gin}{width=\linewidth,totalheight=\textheight,keepaspectratio}
\graphicspath{{.}}

\title{Notes for ECE 30500 - Semiconductor Devices}
\author[Shubham Saluja Kumar Agarwal]{Shubham Saluja Kumar Agarwal}
\date{\today}  % if the \date{} command is left out, the current date will be used

% The following package makes prettier tables.  We're all about the bling!
\usepackage{booktabs}

% The fancyvrb package lets us customize the formatting of verbatim
% environments.  We use a slightly smaller font.
\usepackage{fancyvrb}
\fvset{fontsize=\normalsize}

% Small sections of multiple columns
\usepackage{multicol}

% These commands are used to pretty-print LaTeX commands
\newcommand{\doccmd}[1]{\texttt{\textbackslash#1}}% command name -- adds backslash automatically
\newcommand{\docopt}[1]{\ensuremath{\langle}\textrm{\textit{#1}}\ensuremath{\rangle}}% optional command argument
\newcommand{\docarg}[1]{\textrm{\textit{#1}}}% (required) command argument
\newenvironment{docspec}{\begin{quote}\noindent}{\end{quote}}% command specification environment
\newcommand{\docenv}[1]{\textsf{#1}}% environment name
\newcommand{\docpkg}[1]{\texttt{#1}}% package name
\newcommand{\doccls}[1]{\texttt{#1}}% document class name
\newcommand{\docclsopt}[1]{\texttt{#1}}% document class option name

% Define a custom command for definitions
\newcommand{\defn}[2]{\noindent\textbf{#1}:\ #2}

\begin{document}

\maketitle

\begin{abstract}
    These are lecture notes for Fall 2025 ECE 30500 by professor Elliott at Purdue. Modify, use, and distribute as you please.
\end{abstract}

\tableofcontents

\newpage

\section{Gradient, Divergence, and Curl}
\subsection{Gradient}
The gradient describes the spatial slope of a 3-dimensional function. It can only be applied to a scalar field.\\~\\
Rectangular:
\begin{equation*}
    \nabla f = a_x \frac{\delta d}{\delta x}+a_y \frac{\delta d}{\delta y}+a_z \frac{\delta d}{\delta z}
\end{equation*}
Cylindrical:
\begin{equation*}
    \nabla f = a_\rho \frac{\delta d}{\delta \rho}+a_\phi \frac{1}{\rho}\frac{\delta d}{\delta \phi}+a_z \frac{\delta d}{\delta z}
\end{equation*}
Spherical:
\begin{equation*}
    \nabla f = a_R \frac{\delta d}{\delta R}+a_\theta \frac{1}{R}\frac{\delta d}{\delta \theta}+a_\phi \frac{1}{R\sin(\theta)}\frac{\delta d}{\delta \phi}
\end{equation*}
\subsection{Divergence}
Describes the rate of change of a vector function.\\~\\
Rectangular:
\begin{equation*}
    \nabla \cdot D = \left(a_x \frac{\delta}{\delta x}+a_y \frac{\delta}{\delta y}+a_z \frac{\delta}{\delta z}\right)\cdot \left(a_x D_x +a_y D_y + a_z D_z\right) = \frac{\delta D_x}{\delta x}+\frac{\delta D_y}{\delta y}+\frac{\delta D_z}{\delta z}
\end{equation*}
\subsection{Curl}
Describes the rotation of a vector function.\\
Rectangular:
\begin{equation*}
    \nabla \times D = a_x\left(\frac{\delta D_z}{\delta y} - \frac{\delta D_y}{\delta z}\right)+ a_y\left(\frac{\delta D_x}{\delta z} - \frac{\delta D_z}{\delta x}\right)+ a_z\left(\frac{\delta D_y}{\delta x} - \frac{\delta D_x}{\delta y}\right)
\end{equation*}
Cylindrical:
\begin{equation*}
    \nabla \times D = a_\rho\left(\frac{\delta D_z}{\delta \phi} - \frac{\delta \rho D_\phi}{\delta z}\right)+ \rho a_\phi\left(\frac{\delta D_\rho}{\delta z} - \frac{\delta D_z}{\delta \rho}\right)+ a_z \left(\frac{\delta \rho D_\phi}{\delta \rho} - \frac{\delta D_\rho}{\delta \phi}\right)
\end{equation*}
\subsection{Identities}
\begin{enumerate}
    \item $\nabla \times \nabla V = 0$: the gradient does not rotate.
    \item $\nabla \cdot (\nabla \times A) = 0$: the curl of a vector function does not diverge (grow).
    \item A vector field whose divergence and curl are known is completely determined.
\end{enumerate}
\end{document}
