\documentclass[nobib]{tufte-handout}

%\\geometry{showframe}% for debugging purposes -- displays the margins

\newcommand{\bra}[1]{\left(#1\right)}
\usepackage{hyperref}
\usepackage[activate={true,nocompatibility},final,tracking=true,kerning=true,spacing=true,factor=1100,stretch=10,shrink=10]{microtype}
\usepackage{color}

% Fixes captions and images being cut off
\usepackage{marginfix}
\usepackage{enumitem}
\usepackage{pgfplots}
\usepackage[americancurrents, americaninductors, americanvoltages, americanresistors]{circuitikz}
\usepackage{siunitx}
\usepackage{amsmath,amsthm}
\usetikzlibrary{shapes}
\usetikzlibrary{positioning}
\usetikzlibrary{arrows}

% Set up the images/graphics package
\usepackage{graphicx}
\setkeys{Gin}{width=\linewidth,totalheight=\textheight,keepaspectratio}
\graphicspath{{.}}

\title{Notes for ECE 30100 - Signals and Systems}
\author[Shubham Saluja Kumar Agarwal]{Shubham Saluja Kumar Agarwal}
\date{\today}  % if the \date{} command is left out, the current date will be used

% The following package makes prettier tables.  We're all about the bling!
\usepackage{booktabs}
\usepackage{mdframed}
% The fancyvrb package lets us customize the formatting of verbatim
% environments.  We use a slightly smaller font.
\usepackage{fancyvrb}
\fvset{fontsize=\normalsize}

% Small sections of multiple columns
\usepackage{multicol}

% These commands are used to pretty-print LaTeX commands
\newcommand{\doccmd}[1]{\texttt{\textbackslash#1}}% command name -- adds backslash automatically
\newcommand{\docopt}[1]{\ensuremath{\langle}\textrm{\textit{#1}}\ensuremath{\rangle}}% optional command argument
\newcommand{\docarg}[1]{\textrm{\textit{#1}}}% (required) command argument
\newenvironment{docspec}{\begin{quote}\noindent}{\end{quote}}% command specification environment
\newcommand{\docenv}[1]{\textsf{#1}}% environment name
\newcommand{\docpkg}[1]{\texttt{#1}}% package name
\newcommand{\doccls}[1]{\texttt{#1}}% document class name
\newcommand{\docclsopt}[1]{\texttt{#1}}% document class option name

% Define a custom command for definitions
\newmdenv[
    backgroundcolor=black!10, % Light blue shading inside the box
    linecolor=black,         % Border color
    linewidth=1pt,          % Border thickness
    roundcorner=5pt,        % Rounded corners
    skipabove=10pt,         % Space above the box
    skipbelow=10pt,         % Space below the box
    innertopmargin=5pt,     % Inner top margin
    innerbottommargin=5pt,  % Inner bottom margin
    innerleftmargin=10pt,   % Inner left margin
    innerrightmargin=10pt   % Inner right margin
    framemethod=tikz % Disable problematic referencing
    ]{defbox}

\newmdenv[
    backgroundcolor=cyan!10, % Light blue shading inside the box
    linecolor=cyan,         % Border color
    linewidth=1pt,          % Border thickness
    roundcorner=5pt,        % Rounded corners
    skipabove=10pt,         % Space above the box
    skipbelow=10pt,         % Space below the box
    innertopmargin=5pt,     % Inner top margin
    innerbottommargin=5pt,  % Inner bottom margin
    innerleftmargin=10pt,   % Inner left margin
    innerrightmargin=10pt   % Inner right margin
    framemethod=tikz % Disable problematic referencing
    ]{notebox}
    
\newcommand{\defn}[2]{
        \begin{defbox}
        \noindent\textbf{#1}:\ #2
        \end{defbox}
}
\newcommand{\note}[1]{
        \begin{notebox}
        \noindent\textit{Note}:\ #1
        \end{notebox}
}

\begin{document}

\maketitle

\begin{abstract}
    These are lecture notes for Fall 2025 ECE 30100 at Purdue. Modify, use, and distribute as you please.
\end{abstract}

\tableofcontents

\newpage

\section{Introduction}
A signal can be continuous time (CT) signal, which has an independent
continuous variable indexed $t\in \mathbb{R}$, or discrete time (DT) which has
a discrete independent variable indexed $n\in \mathbb{N}(\mathbb{Z})$.\\

$(.) \rightarrow CT$ \\
$[.] \rightarrow DT$\\~\\

A system, on the other hand, is something that transforms inputs into outputs.
\begin{equation*}
    input \rightarrow [\mathbf{SYSTEM}] \rightarrow output
\end{equation*}
Another way this could be represented is:
\begin{equation*}
    system(input,t) = output
\end{equation*}

These can also be divided into CT and DT.\\ A CT system is of the form:
\begin{equation*}
    x(t) \rightarrow [CT] \rightarrow y(t)
\end{equation*}
On the other hand, a DT is of the form:
\begin{equation*}
    x[n] \rightarrow [DT] \rightarrow y[n]
\end{equation*}
\note{ For most of the course, continuous and discrete will be analyzed separately. That is, only a couple topics will have CT inputs with DT outputs, or DT inputs with CT outputs.}
\note{ Most of the analyzed systems we will be linear and time invariant.}

\section{Linearity}
A system is linear if superposition holds. That is, if it can be analyzed by
analyzing the individual components of the system and combining it.\\ A more
formal definition would be "given an input, which can be represented as the
weighted sum of several inputs, the output can be represented as the sum of
several weighted outputs".\\ The necessary and sufficient conditions for
linearity are:
\begin{itemize}
    \item CT: $\alpha_1 x_1(t)+\alpha_2 x_2(t)\cdots\xrightarrow[]{S} \alpha_1
              y_1(t)+\alpha_2 y_2(t)\cdots$
    \item DT: $\alpha_1 x_1[n]+\alpha_2 x_2[n]\cdots\xrightarrow[]{S} \alpha_1
              y_1[n]+\alpha_2 y_2[n]\cdots$
\end{itemize}
for all values of $\alpha$.\\

Linearity gives us an alternative way to represent and analyze a system. That
is, if we know the responses of all the subcomponents of the input, we can
calculate the response of the input without having to calculate it for the
input directly.
\section{Signal Classification}
\begin{enumerate}
    \item DT vs. CT:\begin{itemize}
              \item DT: $x[n]$ is a sequence of either real or complex valued numbers. $x[n]$ can
                    be written as $x_{Re}[n]+jx_{Im}[n]$ or as $A[n]e^{j\phi[n]}$.\\ We can
                    transform between the two notations using Euler's formula:
                    \begin{equation*}
                        x[n] = A[n]e^{j\phi[n]} = A[n]\cos(\phi[n])+jA[n]\sin(\phi[n])
                    \end{equation*}
              \item CT: $x(t)$ behaves similarly, with the only difference being that it is in
                    terms of $t$. Euler's formula still applies.
          \end{itemize}
    \item Energy and Power: \begin{itemize}
              \item Energy is the area under the squared magnitude of the signal, and it represents
                    how costly it is to store and/or transmit the signal. \\For CT signals, energy
                    over times $t \in (t_1,t_2)$ is \begin{equation*}
                        E = \int_{t_1}^{t_2} |x(t)|^2 dt = \int_{t_1}^{t_2}(x_{Re}^2(t)+jx_{Im}^2(t)) dt
                    \end{equation*}
                    For DT signals, energy over $n \in [n_1,n_2]$ is:
                    \begin{equation*}
                        E = \sum_{n=n_1}^{n_2}|x[n]|^2 = \sum_{n=n_1}^{n_2}(x_{Re}^2[n]+jx_{Im}^2[n])
                    \end{equation*}
                    The total energy can be written as: \begin{equation*}
                        E_\infty = \int_{-\infty}^{\infty} |x(t)|^2 dt = \sum_{n = -\infty}^{\infty} |x[n]|^2
                    \end{equation*}
              \item Power: \\
                    For CT signals, average power over $(t_1,t_2)$ is \begin{equation*}
                        P = \frac{1}{t_2-t_1}E(t_1,t_2)
                    \end{equation*}
                    For DT signals:
                    \begin{equation*}
                        P = \frac{1}{n_2-n_1+1}E[n_1,n_2]
                    \end{equation*}
                    And the overall average power will be:\begin{equation*}
                        P_\infty = \lim_{T\rightarrow\infty}\frac{1}{2T}\int_{-T}^{T}|x(t)|^2 dt = \lim_{N\rightarrow\infty}\frac{1}{2N+1}\sum_{n=-N}^{N}|x[n]|^2
                    \end{equation*}
                    And finally, instantaneous power is $|x(t)|^2$ or $|x[n]|^2$
          \end{itemize}
          There are three realistic types of signals:\begin{enumerate}
            \item $E_\infty$ finite: Must have $P_\infty =0$
            \item $P_\infty$ finite: Must have $e_\infty = \infty$, since we are integrating over time.
            \item Neither $E_\infty$ nor $P_\infty$ are finite: not practical, but mathematically possible.
          \end{enumerate}
          \note{ We cannot have finite energy, $\infty$ power signals.}
          
\end{enumerate}
\section{Transformations in Time}
\begin{enumerate}
    \item Time Shift:
    \begin{align*}
        x(t) \rightarrow &[TS] \rightarrow x(t-t_0)\\
        x[n] \rightarrow &[TS] \rightarrow x[n-n_0]
    \end{align*}
    \begin{itemize}
        \item $t_0,n_0>0$: signal is shifted to the right, or delayed by $t_0$
        \item $t_0, n_0<0$: signal is shifted to the left, or advanced by $t_0$
    \end{itemize}
    \item Time Reversal:
    \begin{align*}
        x(t)\rightarrow &[TR]\rightarrow x(-t)\\
        x[n]\rightarrow &[TR]\rightarrow x[-n]\\
    \end{align*}
    \item Time Scaling:
    \begin{align*}
        x(t)\rightarrow &[TSc]\rightarrow x(\alpha t)\\
        x[n]\rightarrow &[TSc]\rightarrow x[c n]
    \end{align*}
    Assume $\alpha, c >0$. If they corresponded to something negative, it can be considered as a scaling and a reversal together.\\
    \note{ $c$ must be a value such that $cn$ is an integer. We are also assured of the fact that we will be losing information for any value of $c\neq 1$.}
    \begin{itemize}
        \item $\alpha, c >1$: shorter timescale, or faster
        \item $\alpha, c <1$: longer timescale, or slower
    \end{itemize}
    \item Composite Transformation: a transformation that involves multiple transformations from 1., 2., 3.\\
    For example:
    \begin{align*}
        x(t)\rightarrow[CT]\rightarrow x(-\alpha t + \beta)
    \end{align*}
    Usually, to decompose these transformations, we shift, then scale, then reverse, as and if necessary.
\end{enumerate}
\section{More Signal Classifications}
\begin{enumerate}[start = 3]
    \item By the period (symmetry under time shift)\\
    CT signal is periodic $\iff \exists T\neq 0 | x(t)=x(t+T) \forall T$. That signal is periodic with period $t$.\\
    DT signal is periodic $\iff \exists N\neq 0 | x[n]=x[n+N] \forall N$. That signal is periodic with period $n$.
\end{enumerate}
\end{document}
