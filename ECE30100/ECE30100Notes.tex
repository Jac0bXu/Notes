\documentclass[nobib]{tufte-handout}

%\\geometry{showframe}% for debugging purposes -- displays the margins

\newcommand{\bra}[1]{\left(#1\right)}
\usepackage{hyperref}
\usepackage[activate={true,nocompatibility},final,tracking=true,kerning=true,spacing=true,factor=1100,stretch=10,shrink=10]{microtype}
\usepackage{color}

% Fixes captions and images being cut off
\usepackage{marginfix}

\usepackage{pgfplots}
\usepackage[americancurrents, americaninductors, americanvoltages, americanresistors]{circuitikz}
\usepackage{siunitx}
\usepackage{amsmath,amsthm}
\usetikzlibrary{shapes}
\usetikzlibrary{positioning}
\usetikzlibrary{arrows}

% Set up the images/graphics package
\usepackage{graphicx}
\setkeys{Gin}{width=\linewidth,totalheight=\textheight,keepaspectratio}
\graphicspath{{.}}

\title{Notes for ECE 30100 - Signals and Systems}
\author[Shubham Saluja Kumar Agarwal]{Shubham Saluja Kumar Agarwal}
\date{\today}  % if the \date{} command is left out, the current date will be used

% The following package makes prettier tables.  We're all about the bling!
\usepackage{booktabs}

% The fancyvrb package lets us customize the formatting of verbatim
% environments.  We use a slightly smaller font.
\usepackage{fancyvrb}
\fvset{fontsize=\normalsize}

% Small sections of multiple columns
\usepackage{multicol}

% These commands are used to pretty-print LaTeX commands
\newcommand{\doccmd}[1]{\texttt{\textbackslash#1}}% command name -- adds backslash automatically
\newcommand{\docopt}[1]{\ensuremath{\langle}\textrm{\textit{#1}}\ensuremath{\rangle}}% optional command argument
\newcommand{\docarg}[1]{\textrm{\textit{#1}}}% (required) command argument
\newenvironment{docspec}{\begin{quote}\noindent}{\end{quote}}% command specification environment
\newcommand{\docenv}[1]{\textsf{#1}}% environment name
\newcommand{\docpkg}[1]{\texttt{#1}}% package name
\newcommand{\doccls}[1]{\texttt{#1}}% document class name
\newcommand{\docclsopt}[1]{\texttt{#1}}% document class option name

% Define a custom command for definitions
\newcommand{\defn}[2]{\noindent\textbf{#1}:\ #2}

\begin{document}

\maketitle

\begin{abstract}
    These are lecture notes for Fall 2025 ECE 30100 at Purdue. Modify, use, and distribute as you please.
\end{abstract}

\tableofcontents

\newpage

\section{Introduction}
A signal can be continuous time (CT) signal, which has an independent
continuous variable indexed $t\in \mathbb{R}$, or discrete time (DT) which has
a discrete independent variable indexed $n\in \mathbb{N}(\mathbb{Z})$.\\

$(.) \rightarrow CT$ \\
$[.] \rightarrow DT$\\~\\

A system, on the other hand, is something that transforms inputs into outputs.
\begin{equation*}
    input \rightarrow [\mathbf{SYSTEM}] \rightarrow output
\end{equation*}
Another way this could be represented is:
\begin{equation*}
    system(input,t) = output
\end{equation*}

These can also be divided into CT and DT.\\ A CT system is of the form:
\begin{equation*}
    x(t) \rightarrow [CT] \rightarrow y(t)
\end{equation*}
On the other hand, a DT is of the form:
\begin{equation*}
    x[n] \rightarrow [DT] \rightarrow y[n]
\end{equation*}
\textit{Note : For most of the course, continuous and discrete will be analyzed separately. That is, only a couple topics will have CT inputs with DT outputs, or DT inputs with CT outputs.}\\
\textit{Note : Most of the analyzed systems we will be linear and time invariant.}\\

\section{Linearity}
A system is linear if superposition holds. That is, if it can be analyzed by analyzing the individual components of the system and combining it.\\
A more formal definition would be "given an input, which can be represented as the weighted sum of several inputs, the output can be represented as the sum of several weighted outputs".\\
The necessary and sufficient conditions for linearity are:
\begin{itemize}
    \item CT: $\alpha_1 x_1(t)+\alpha_2 x_2(t)\cdots\xrightarrow[]{S} \alpha_1 y_1(t)+\alpha_2 y_2(t)\cdots$
    \item DT: $\alpha_1 x_1[n]+\alpha_2 x_2[n]\cdots\xrightarrow[]{S} \alpha_1 y_1[n]+\alpha_2 y_2[n]\cdots$
\end{itemize}
for all values of $\alpha$.\\

Linearity gives us an alternative way to represent and analyze a system. That is, if we know the responses of all the subcomponents of the input, we can calculate the response of the input without having to calculate it for the input directly.

\end{document}
