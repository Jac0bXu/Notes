\documentclass[nobib]{tufte-handout}

%\\geometry{showframe}% for debugging purposes -- displays the margins

\newcommand{\bra}[1]{\left(#1\right)}
\usepackage{amssymb}
\usepackage{hyperref}
\usepackage{pgfplots}
\usepackage[activate={true,nocompatibility},final,tracking=true,kerning=true,spacing=true,factor=1100,stretch=10,shrink=10]{microtype}
\usepackage{color}
\usepackage{steinmetz}
% Fixes captions and images being cut off
\usepackage{marginfix}
\usepackage{array}
\usepackage{tikz}
\usepackage{amsmath,amsthm}
\usetikzlibrary{shapes}
\usetikzlibrary{positioning}
\usepackage{listings}
\usepackage{caption}
\DeclareCaptionFont{white}{\color{white}}
\DeclareCaptionFormat{listing}{\colorbox{gray}{\parbox{\textwidth}{#1#2#3}}}
\captionsetup[lstlisting]{format=listing,labelfont=white,textfont=white}

% Set up the images/graphics package
\usepackage{graphicx}
\setkeys{Gin}{width=\linewidth,totalheight=\textheight,keepaspectratio}
\graphicspath{{.}}

\title{Notes for ECON 25100 - Microeconomics}
\author[Shubham Saluja Kumar Agarwal]{Shubham Saluja Kumar Agarwal}
\date{\today}  % if the \date{} command is left out, the current date will be used

% The following package makes prettier tables.  We're all about the bling!
\usepackage{booktabs}

% The units package provides nice, non-stacked fractions and better spacing
% for units.
\usepackage{units}

% The fancyvrb package lets us customize the formatting of verbatim
% environments.  We use a slightly smaller font.
\usepackage{fancyvrb}
\fvset{fontsize=\normalsize}

% Small sections of multiple columns
\usepackage{multicol}

% For finite state machines 
\usetikzlibrary{automata} % Import library for drawing automata
\usetikzlibrary{positioning} % ...positioning nodes
\usetikzlibrary{arrows} % ...customizing arrows
\tikzset{node distance=2.5cm, % Minimum distance between two nodes. Change if necessary.
    every state/.style={ % Sets the properties for each state
    semithick,
    fill=gray!10},
    initial text={}, % No label on start arrow
    double distance=2pt, % Adjust appearance of accept states
    every edge/.style={ % Sets the properties for each transition
    draw,
    ->,>=stealth', % Makes edges directed with bold arrowheads
    auto,
    semithick}}
\let\epsilon\varepsilon

% These commands are used to pretty-print LaTeX commands
\newcommand{\doccmd}[1]{\texttt{\textbackslash#1}}% command name -- adds backslash automatically
\newcommand{\docopt}[1]{\ensuremath{\langle}\textrm{\textit{#1}}\ensuremath{\rangle}}% optional command argument
\newcommand{\docarg}[1]{\textrm{\textit{#1}}}% (required) command argument
\newenvironment{docspec}{\begin{quote}\noindent}{\end{quote}}% command specification environment
\newcommand{\docenv}[1]{\textsf{#1}}% environment name
\newcommand{\docpkg}[1]{\texttt{#1}}% package name
\newcommand{\doccls}[1]{\texttt{#1}}% document class name
\newcommand{\docclsopt}[1]{\texttt{#1}}% document class option name

% Define a custom command for definitions and biconditional
\newcommand{\defn}[2]{\noindent\textbf{#1}:\ #2}
\let\biconditional\leftrightarrow

\begin{document}

\maketitle

\begin{abstract}
    These are lecture notes for spring 2024 ECON 25100 at Purdue. Modify, use, and distribute as you please.
\end{abstract}

\tableofcontents

\section{Course Introduction}
This course offers a comprehensive exploration of the principles that govern
individual economic decision-making and the interactions within markets.

\pagebreak

\section{Introduction to Economic Principles}
\textit{This section will briefly define several important terms, topics,and principles relevant to the rest of this class.}\\~\\
Economics is the study of allocation of scarce resources to meet the unlimited
human wants.
\begin{itemize}
    \item Microeconomics: decision making by individual economic agents such as firms and
          consumers.
    \item Macroeconomics: aggregate performance of the entire economic system.
    \item Emipirical economics: facts to present a description of economic activity.
    \item Economic theory: relies upon principles to analyze behavior of economic agents.
\end{itemize}
Economy has slowly transitioned from mainly theoretical to mainly empirical.\\
Assumptions are made consistently, sos as to more rigurously create a methodology to analyze the world without overly complicating things.\\
Model building is the creation of abstractions from reality.\\
\quad Occam's razor: The best model is that which describes reality and is the simplest.\\
\quad "Ceteris Paribus": All other things equal. (changing only certain parameters and leaving everything else the same)\\
\quad The lack of assumptions would make things either to simple or too complicated to viably describe reality.\\
\quad Economics provides a method to make a rational choice.\\
\quad Rigurous models are made to predict human behavior through either inductive logic, or deductive logic.\\
There are two kinds of economics:
\begin{itemize}
    \item Positive Economics: concerned with reality.
    \item Normative economics: concerned with what should be. (If a statement has
          "should" it's probably normative).
\end{itemize}
\quad The economic problem involves the allocation or resources among comepting wants.\\
\quad This exists due to scarcity.\\
\quad Scarcity exists because of unlimited human wants and limited resources.\\~\\
Economic Resources:\\
\begin{itemize}
    \item Land - space, natural resources.
    \item Capital - physical assets like factories or tractors.
    \item Labor - skills, abilities, knowledge, etc.
    \item Entrepeneurial talent - the economic agent who creates the enterprise.
    \item Technology - a manner in which resources are combined to produce commodities
          (methods of making processes more efficient).
\end{itemize}
Core Principles of Economics:
\begin{itemize}
    \item Cost-Benefit Principle: cost and benefits are the incentives. Do something if
          the benefits outweight the costs. (Convert everything to money, and then
          calculate).\\ The cost-benefit principle is directly related to the willingness
          to pay, which is precisely, the conversion of benefits to money.\\ If the
          benefit is greater than the cost, one has achieved an economic surplus. This
          principle aims to maximize the economic surplus.\\ It also relates to framing
          effects, which is when a decision is affected by the method in which the
          situation or object is framed.
    \item Opportunity Cost Principle:
    \item Marginal Principle:
    \item Interdependence Principle:
\end{itemize}
\end{document}