\documentclass[nobib,fleqn,8pt]{article}

%\\geometry{showframe}% for debugging purposes -- displays the margins
\usepackage{geometry}
\geometry{margin=0.5in} % Example of setting 1-inch margins

\newcommand{\bra}[1]{\left(#1\right)}
\usepackage{amssymb}
\usepackage{hyperref}
\usepackage[activate={true,nocompatibility},final,tracking=true,kerning=true,spacing=true,factor=1100,stretch=10,shrink=10]{microtype}
\usepackage{color}
\usepackage{steinmetz}
% Fixes captions and images being cut off
\usepackage{marginfix}
\usepackage{array}
\usepackage{tikz}
\usepackage{amsmath,amsthm}
\usetikzlibrary{shapes}
\usetikzlibrary{positioning}
\usepackage{listings}
\usepackage{caption}
\usepackage{circuitikz}
\DeclareCaptionFont{white}{\color{white}}
\DeclareCaptionFormat{listing}{\colorbox{gray}{\parbox{\textwidth}{#1#2#3}}}
\captionsetup[lstlisting]{format=listing,labelfont=white,textfont=white}

% Set up the images/graphics package
\usepackage{graphicx}
\setkeys{Gin}{width=\linewidth,totalheight=\textheight,keepaspectratio}
\graphicspath{{.}}
\usepackage{pgfplots}
% The following package makes prettier tables.  We're all about the bling!
\usepackage{booktabs}

% The units package provides nice, non-stacked fractions and better spacing
% for units.
\usepackage{units}

% The fancyvrb package lets us customize the formatting of verbatim
% environments.  We use a slightly smaller font.
\usepackage{fancyvrb}
\fvset{fontsize=\small}

% Small sections of multiple columns
\usepackage{multicol}


% These commands are used to pretty-print LaTeX commands
\newcommand{\doccmd}[1]{\texttt{\textbackslash#1}}% command name -- adds backslash automatically
\newcommand{\docopt}[1]{\ensuremath{\langle}\textrm{\textit{#1}}\ensuremath{\rangle}}% optional command argument
\newcommand{\docarg}[1]{\textrm{\textit{#1}}}% (required) command argument
\newenvironment{docspec}{\begin{quote}\noindent}{\end{quote}}% command specification environment
\newcommand{\docenv}[1]{\textsf{#1}}% environment name
\newcommand{\docpkg}[1]{\texttt{#1}}% package name
\newcommand{\doccls}[1]{\texttt{#1}}% document class name
\newcommand{\docclsopt}[1]{\texttt{#1}}% document class option name

% Define a custom command for definitions and biconditional
\newcommand{\defn}[2]{\noindent\textbf{#1}:\ #2}
\let\biconditional\leftrightarrow
\linespread{0.9}
\setlength{\parindent}{0pt}

\begin{document}
$MR=MC$\\
$P=f(Q)$\\~\\
Current Price of Investments:
\begin{equation*}
    PV_1 = \frac{FV}{(1+i)^n}
\end{equation*}
If it generates a value at the end of each year:
\begin{equation*}
    PV_{stream} = \sum_{t=1}^n\frac{FV_t}{(1+i)^t}
\end{equation*}
\begin{equation*}
    PV_{net} = \sum_{t=1}^n\frac{FV_t}{(1+i)^t}-C_0 \text{ -where $C_0$ is the current investment}
\end{equation*}
Current value of a firm before paying
\begin{equation*}
    PV_{firm} = \pi_0\frac{i+1}{i-g}
\end{equation*}
Current value of a firm after paying
\begin{equation*}
    PV_{firm} = \pi_0\frac{i+1}{i-g} - \pi_0 = \frac{\pi_0(1+g)}{i-g}
\end{equation*}
\begin{equation*}
    PV_{perpetuity} = \frac{CF}{i}
\end{equation*}
Full Economic Price: Amount consumers would be willing to pay if the supply was the amount being sold under a price ceiling.\\
Deadweight loss when surplus is triangle on the right of intersection.\\
Deadweight loss when shortage is triangle on the left of intersection.\\
Deadweight loss when govt. pays is \textbf{M} shaped.\\~\\
Least squares regression: $y=\hat{a}+\hat{b}x$\\
CI: $\hat{a}\pm 2\sigma_{\hat{a}}$,$\hat{b}\pm 2\sigma_{\hat{b}}$ where $\sigma$ are standard errors.\\
The t-stat is: $t_{\hat{x}}=\frac{\hat{x}}{\sigma_{\hat{x}}}$ where $x=a,b$\\
\begin{equation*}
    Marginal_{revenue} = Price\left(\frac{1+Elasticity}{Elasticity}\right)
\end{equation*}
\begin{equation*}
    Elasticity = \frac{\delta Q_x^d}{\delta P} \frac{P}{Q_x} \textit{ (or M if income)} = \left\vert \frac{\frac{Q_2-Q_1}{(Q_2+Q_1)}}{\frac{P_2-P_1}{(P_2+P_1)}}\right\vert
\end{equation*}
Demand is elastic if the revenue increases by decreasing the price, that is $|E>1|$.\\
Own demand elasticity is affected by substitute availability, time of purchase horizon, expenditure of consumer budgets.\\
In cross-product, if $E<0$, products are complements. In income, if $E<0$, product is inferior.\\
If Q function is logarithmic, coefficients are elasticities.\\
Marginal rate of substitution = $|\frac{d}{dQ_x}Q_y|$\\
If a firms revenues are derived from the sales of two products, y and x:
\begin{equation*}
    \Delta Revenue = [Rev_x(1+E_{xx})+Rev_y E_{yx}]\times \%\Delta P_x
\end{equation*}
For producer:
\begin{equation*}
    Market_{subs_{rate}}(x,y) = -\frac{P_x}{P_y} \text{ otherwise, equilibrium is not achieved}
\end{equation*}
For consumer:
\begin{equation*}
    Marginal_{rate_{subs}}(x,y) =MRS_{x,y} =  \frac{P_x}{P_y} \text{ otherwise,equilibrium consumption bundle is not achieved}
\end{equation*}
\begin{eqnarray*}
    MRTS_{KL} = \frac{MP_L}{MP_K} ( = \frac{\mathbf{w}age}{\mathbf{r}ent} \text{ for cost minimization} = tangent(isoquant=isocost))\\
    \text{remember that } K\uparrow \implies MP_K \downarrow\\
    Value_{Marginal_{Product}}= VMP_x = P_{rice}\times MP_x (= w \text{ if $x=L_{abor}$, fixed }K_{apital})
\end{eqnarray*}
\begin{eqnarray*}
    C(Q_1,0)+C(0,Q_2)>C(Q_1,Q_2) \implies \text{Economy of Scope}\\
    \frac{\Delta MC_1(Q_1,Q_2)}{\Delta Q_2} = \frac{\delta}{\delta Q_2}(\frac{\delta}{\delta Q_1} C(Q_1,Q_2))< 0 \implies \text{Cost Complementarity}
\end{eqnarray*}
\begin{eqnarray*}
    \frac{d}{dQ}AVG_{TOT_{cost}}<0 \implies \text{economy of scale}
\end{eqnarray*}
\begin{itemize}
    \item Contract - well defined and measurable and highly specialized with optimal
          length $L^* = L(MB=MC)$
    \item Specialized Investment - exchange that cannot be recovered, relationship
          specific (site, physical asset, dedicated assets, human capital). Leads to
          costly bargaining, underinvestment, opportunism and hold up.
    \item Spot exchange - modest number,standardized,many sellers,opportunism not
          problem,underinvestment$\implies$bad quality
    \item Vertical integration - buying internally, or from subsidiaries, less costs and
          opportunism, more work.
    \item hold-up problem - losses from contract breach
    \item principal-agent problem - agent/principal goals don't align (conflict of
          interest), solved using \%profits or spot checks
\end{itemize}
\begin{equation*}
    NewContractValue = (\sum SunkCosts) (-1?)
\end{equation*}
\begin{equation*}
    4FirmConcentrationRatio = \frac{F1+F2+F3+F4}{TotalMarket}
\end{equation*}
\begin{eqnarray*}
    HHI = 10000\times\sum_i\left(\frac{F_i}{TotalMarket}\right)^2 \\
    \text{block if merger makes $HHI_n>2500$ or $\Delta HHI > 200$}
\end{eqnarray*}
\begin{equation*}
    RothschildIndex = \frac{OwnElaticity}{DemandElasticity}
\end{equation*}
\begin{eqnarray*}
    P = \frac{1}{1-LernerIndex}\times MC\\
    MarkUpFactor = \frac{P}{MC}
\end{eqnarray*}
Perfect Competition:
\begin{itemize}
    \item Many buyers and sellers
    \item homogenous products
    \item perfect information in buyers and sellers
    \item free entry and exit
\end{itemize}
Market Parameters
\begin{itemize}
    \item Monopoly: Rothschild = 1, HHI=10000, C4=1
    \item Oligopoly: Rothschild = large, HHI = large, C4 =1
    \item Monopolistic Competition: Rothschild = small $>>0$, HHI = small, C4 = small
    \item Perfect Competition: Rothschild = very small, HHI = very small, C4 = very small
\end{itemize}

A firm should shut down if $profit<fixed_{cost}$ or $P<AVC$.\\ Entry to a market occurs
until economic profits shrink to 0.\\ Profit Maximizing Price is: $P_{demand}$
at $Q_{MR=MC}$\\ Revenue Maximizing Price is: $P_{demand}$ at $Q_{MR=0}$\\~\\
Let $P = b-mQ$:\\ $MR = \frac{d}{dQ}PQ = b-2mQ = b-2m(Q_1+Q_2) =
    P\left(\frac{1+E}{E}\right)$\\ For a multiple source monopoly: $MR(Q_1,Q_2) =
    MC_1(Q_1)$ and $MR(Q_1,Q_2) = MC_2(Q_2)$\\~\\ Slope in oligopoly where opposing
firm does not match changes is more horizontal.\\ Range of marginal cost to
keep price: $[MR(D_1(P)),MR(D_2(P))]$\\~\\ 
In a Sweezy oligopoly:
Few firms, many customers, differentiated products, rivals will cut in response to cut, but not increase in response to increase.\\
In a Cournot duopoly:
Few firms, many customers, any kind of product, rival holds constant output in response to changes.
\begin{eqnarray*}
    P=a-b(Q_1+Q_2)\\
    \implies MR_1(Q_1,Q2) = a-bQ_2-2bQ_1, MR_2(Q_1,Q2) = a-bQ_1-2bQ_2\\
    \implies Q_1 = r_1(Q_2)= \frac{a-c_1}{2b}-\frac{1}{2}Q_2,Q_2 = r_2(Q_1)= \frac{a-c_2}{2b}-\frac{1}{2}Q_1
\end{eqnarray*}
In a Stackelberg duopoly: few firms, many customers, any kind of product, one leader defines price before, all others follow 
\begin{eqnarray*}
    P = a-b(Q_L+Q_F)\\
    \implies Q_L = \frac{a+c_F-2c_L}{2b}, Q_F= \frac{a-c_F}{2b}-\frac{1}{2}Q_L
\end{eqnarray*}
In a Bertrand oligopoly: few firms, many customers, identical products, constant MC, price comepetition, perfect info, no transaction cost
\begin{eqnarray*}
    P = a-bQ, P_{eq} = MC\\
    Q = \frac{a-MC}{b}\\
    \pi = 0
\end{eqnarray*}
In Collusive Environments, which behave like monopolies:
\begin{equation*}
    MR = MC \implies Q = \frac{a-MC}{2b}
\end{equation*}
Contestable market if: same tech, consumer quick price change response, firms slow price change response, no sunk cost.\\~\\
In a one-shot game: the strategy is the one with the best payout for each of the other player's choices.\\
In a game where the one-shot Nash equilibrium points at nonideal payout, \textbf{near-infinite} plays are necessary to earn more. If it is an infinite game with interest, higher payoff is possible if:
\begin{equation*}
    \frac{\pi_{cheat}-\pi_{cooperation}}{\pi_{cooperation}-\pi_{Nash}}\leq \frac{1}{i}
\end{equation*}
If there is a probability $\theta$ for the game to end, no reason to cheat if: $\Pi_A^{Cheat}\leq \frac{\pi_A^{coop}}{\theta} = \Pi_A^{coop}$\\
if it is a multistage game, make a choice tree.\\
In a Monopoly/Mon Comp: $P = \left(\frac{E_F}{1+E_F}\right)MC = K\times MC$ where K is the profit-maximizing markup factor\\
In a Cournot Oligopoly: $P = \left(\frac{NE_M}{1+NE_M}\right)MC = \left(\frac{E_F}{1+E_F}\right)MC$\\~\\
In Price Discrimination:
\begin{itemize}
    \item First Degree: Price sticks to demand curve. $P=D$, $\pi = \int D-MC$
    \item Second Degree: Change price after certain amount is consumed.
    \item Third Degree: $P_1\left(\frac{1+E_1}{E_1}\right)=MC,
              P_2\left(\frac{1+E_2}{E_2}\right)=MC$, $Markup = \frac{E}{E+1}$, needs no
          resale, different E
    \item Two-Part Pricing: $Fixed_{Fee}+PerUnit_{Fee}$ where $Fixed_{Fee} =\pi = \int
              D-MC$ and $PerUnit_{Fee} = MC$
    \item Block Pricing: Sell total quantity as one. $P = \int_{0}^{Q}D$, $\pi = \int
              D-MC$, $Q= Q_{uantity}(D=MC)$
    \item Commodity Bundling: Sell multiple different products as a single bundle.
          Customer will buy if price of bundle is less than sum of individual MB.
    \item Peak-load pricing: higher prices during peak hours
    \item Cross-subsidy: if two products are related through cost, sell one cheaper, the other more expensive
    \item Transfer pricing: internal pricing - upstream set price for lowstream avoiding double marginalization. \\$NRM_D = MR_D-MC_D = MC_U$
    \item Double marginalization: when both upstream and attempt to maximize division profits, instead of overall profits.
    \item Price matching, brand loyalty, randomized pricing are other strategies
\end{itemize}
Random Outcome Probability: $Variance = (Stdev)^2 = \sum P(x)(x-\mu)^2 $, $\mu = Expected_{Value} = \sum x*P(x)$, big stdev is risky\\
\begin{itemize}
    \item Risk Neutral: indifferent to probability vs for sure
    \item Risk Averse: prefers for sure, \textit{this type of customers leads to chain
              stores, online reviews, insurance}
    \item Risk loving: prefers a probability of higher
\end{itemize}
Reservation Price is the price at which $Expected_{benefit} = Search_{Cost}$\\
Diversification: reduce risks by investing in multiple projects\\
Auction:
\begin{itemize}
    \item Independent Private Values:
          \begin{itemize}
              \item First Price, seal: \begin{equation*}
                        b = value-\frac{value-lowest_{valuation}}{num_{bidders}}
                    \end{equation*}
              \item Dutch: let lower until \begin{equation*}
                        b = value-\frac{value-lowest_{valuation}}{num_{bidders}}
                    \end{equation*}
              \item Second Price, seal: $b = v$
              \item English: active until $b=v$
          \end{itemize}
          Expected Revenues for auctioneer: $English = Second-price = first-price = Dutch$
    \item Correlated Values Auctions: avoid winner's curse, which is the fact that the
          winner values it more than all others.\\ Expected revenue for auctioneer:
          $English>Second-price>first-price=Dutch$
\end{itemize}
Assymetric information leads to:
\begin{itemize}
    \item Hidden Actions are morally hazardous, and can be avoided by incentive
          contracts, signaling or screening.
    \item Hidden Characteristics lead to adverse selection, and can be avoided through
          screening and sorting.
\end{itemize}

\end{document}

