\documentclass[nobib]{tufte-handout}

%\\geometry{showframe}% for debugging purposes -- displays the margins

\newcommand{\bra}[1]{\left(#1\right)}
\usepackage{clrscode3e}
\usepackage{hyperref}
\usepackage[activate={true,nocompatibility},final,tracking=true,kerning=true,spacing=true,factor=1100,stretch=10,shrink=10]{microtype}
\usepackage{color}

% Fixes captions and images being cut off
\usepackage{marginfix}

\usepackage{tikz}
\usepackage{amsmath,amsthm}
\usetikzlibrary{shapes}
\usetikzlibrary{positioning}

% Set up the images/graphics package
\usepackage{graphicx}
\setkeys{Gin}{width=\linewidth,totalheight=\textheight,keepaspectratio}
\graphicspath{{.}}

\title{Notes for ECE 369 - Discrete Mathematics for Computer Engineering}
\author[Ezekiel Ulrich]{Ezekiel Ulrich}
\date{\today}  % if the \date{} command is left out, the current date will be used

% The following package makes prettier tables.  We're all about the bling!
\usepackage{booktabs}

% The units package provides nice, non-stacked fractions and better spacing
% for units.
\usepackage{units}

% The fancyvrb package lets us customize the formatting of verbatim
% environments.  We use a slightly smaller font.
\usepackage{fancyvrb}
\fvset{fontsize=\normalsize}

% Small sections of multiple columns
\usepackage{multicol}

% These commands are used to pretty-print LaTeX commands
\newcommand{\doccmd}[1]{\texttt{\textbackslash#1}}% command name -- adds backslash automatically
\newcommand{\docopt}[1]{\ensuremath{\langle}\textrm{\textit{#1}}\ensuremath{\rangle}}% optional command argument
\newcommand{\docarg}[1]{\textrm{\textit{#1}}}% (required) command argument
\newenvironment{docspec}{\begin{quote}\noindent}{\end{quote}}% command specification environment
\newcommand{\docenv}[1]{\textsf{#1}}% environment name
\newcommand{\docpkg}[1]{\texttt{#1}}% package name
\newcommand{\doccls}[1]{\texttt{#1}}% document class name
\newcommand{\docclsopt}[1]{\texttt{#1}}% document class option name

% Define a custom command for definitions
\newcommand{\defn}[2]{\noindent\textbf{#1}:\ #2}

\begin{document}

\maketitle

\begin{abstract}
These are lecture notes for fall 2023 ECE 36900 at Purdue. Modify, use, and distribute as you please.
\end{abstract}

\tableofcontents

\section{Course Introduction}

This course introduces discrete mathematical structures and 
finite-state machines. Students will learn how to use logical 
and mathematical formalisms to formulate and solve problems in 
computer engineering. Topics include formal logic, proof techniques, 
recurrence relations, sets, combinatorics, relations, functions, 
algebraic structures, and finite-state machines. For more information,
see the syllabus. 

\section{Equations}

\begin{enumerate}
    \item De Morgan's Theorem: 
    
    $\neg(P \wedge Q) \equiv \neg P \vee \neg Q$

    $\neg(P \vee Q) \equiv \neg P \wedge \neg Q$
\end{enumerate}

\section{Propositional Logic}

We often wish that others would be more logical, tell the truth,
or shower. While studying formal logic cannot help with the latter
(in fact, studies have shown a negative correlation between
hygiene and studying formal logic) it is a useful way to 
define what the first mean two. In a formal logic model, we have two constructs:
\begin{itemize}
    \item \defn{Statements/propostion}{A statement or a proposition is a sentence that
    is either true or false.} Propositions are often represented 
    with letters of the alphabet. For example: "$q$: the more time
    you spend coding, the less time you have to buy deodorant."
    \item \defn{Logical connectives}{Used to connect statements.} For 
    example, "and" is a logical connective in English. It can be
    used to connect two statements, e.g. "the person next to me
    smells like dog \emph{and} looks like a dog" to obtain 
    a new statement with its own truth value. 
\end{itemize}

Here are common logical connectives in Boolean logic:
\begin{table}[h]
    \centering
    \begin{tabular}{lc}
    \toprule
    \textbf{Logical Connective} & \textbf{Symbol} \\
    \midrule
    Negation (NOT) & $\lnot$ \\
    Conjunction (AND) & $\land$ \\
    Disjunction (OR) & $\lor$ \\
    Exclusive OR (XOR) & $\oplus$ \\
    Implication & $\rightarrow$ \\
    Biconditional & $\leftrightarrow$ \\
    \bottomrule
    \end{tabular}
    \caption{Logical Connectives in Boolean Logic}
    \label{tab:logical-connectives}
\end{table}

\defn{Truth table}{Defines how each of the
connectives operate on truth values.}

\section{Lecture 2}

With each additional variable in your truth table, the number
of choices grows exponentially. Specifically, if you have $n$ statement
letters, you would have $2^n$ choices for your truth table. 

\defn{Tautology}{A formula that is true in every model.} 
Example: the Bible is infallible because the Bible itself
says it is infallible.  

\defn{Contradiction}{A formula that is false in every model} Examples:
"it is raining and it is not raining", "I am sleeping and I am awake", 
"IU is a good school". 

Confusion often arises when negating a sentence such as 
"the book is thick and boring". An natural inclination is to 
negate it thus: "the book is not thick and not boring".
However, consider the truth table for this:
$p$: "the book is thick", $q$: "the book is boring". 
\begin{table}[ht]
    \centering
    \begin{tabular}{|c c|c|c|c|}
    \hline
    $p$ & $q$ & $p \wedge q$ & $\neg(p \wedge q)$ & $\neg p \wedge \neg q$\\
    \hline
    T & T & T & F & F\\
    T & F & F & T & F\\
    F & T & F & T & F\\
    F & F & F & T & T\\
    \hline
    \end{tabular}
    \caption{A Simple Table}
    \label{tab:demorganswrong}
\end{table}
We can see the last two rows are not identical, therefore
the negation of "the book is not thick and not boring"
is not "the book is not thick and not boring". For $p$ to 
be false, either the book must not be thick \emph{or} the
book must not be boring. This is summarized by 
\defn{De Morgan's Theorem}{
    \[\neg(P \wedge Q) \equiv \neg P \vee \neg Q\]
    \[\neg(P \vee Q) \equiv \neg P \wedge \neg Q\]
}

\defn{Modus ponens (mp)}{P, if p then q, therefore q}

\defn{Modus totens (mt)}{if p then q, not q, therefore not p}

%Include equivalence rules table

\end{document}
